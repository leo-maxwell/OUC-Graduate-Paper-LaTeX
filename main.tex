\documentclass[UTF8,zihao=-4]{oucart}
\addbibresource{bibliography/ref.bib}

\title{基于唱跳说唱篮球的舞蹈练习}
\entitle{The Title of the Thesis}
\author{蔡徐坤}
\studentid{123456789}
\advisor{唱跳导师}
\department{信息科学与工程学院}{计算机科学与技术}
\grade{2017}

\cnabstractkeywords{
    出道之后,蔡徐坤大部分精力都投身于新歌的创作和专辑的打造。彼时,他需要随着NINE PERCENT在三个月内完成17场大型巡回见面会,因此写歌的时间必须“挤出来”用。洗澡时、做造型时、飞机上、两个行程间或吃饭的空隙,只要有手机、旋律,任何地方都是他的创作场所;偶尔待在录音室里,甚至成为他的喘息时间。去年,新京报记者见到他时正值午饭,化妆室里传来哼鸣声,“采访完的休息时间,我都可以写一段词。我还年轻,我觉得这都OK。”他曾表示。而《1》的发表同样“违背”偶像市场的规律。蔡徐坤本可以每月发一首,制造更多话题。但他认为,一首首发表并不足以让外界更全面地了解他的音乐风格,“当别人都走得很快,我反而要踏踏实实一步步走。”偶尔听到舆论质疑他没有作品,蔡徐坤也曾犹豫,要不要先发一部分出来?但内心却总有个声音说,“你可以再多做几首不同风格的作品,让大家看到最全面、最好的你,而不是急于求成地去展现自己。”
}{
    蔡徐坤,篮球,舞台
}
\enabstractkeywords{
    After his debut, Cai devoted most of his energy to the creation of new songs and the creation of albums.\ At that time, he needed to complete 17 large\-scale tour meetings with nine percent in three months, so the time for writing songs had to be ``squeezed out".\ When bathing, modeling, on the plane, between two itineraries or between meals, as long as there is a mobile phone and melody, anywhere is his creation place; occasionally stay in the studio, even become his breathing time.\ Last year, when the reporter of the Beijing News saw him, it was lunch time, and there was a hum in the dressing room.\ ``I can write a paragraph during the rest time after the interview.\ I'm still young.\ I think it's OK.\ " He once said.
}{
    Cai Xukun, Basketball, Dance
}

\renewcommand{\baselinestretch}{1.6}

\begin{document}

    \makecover % 封面
    \makesignature % 扉页
    \makeabstract % 摘要

% 目录
    \thispagestyle{tableofcontents}
    \tableofcontents
    \newpage

    \pagenumbering{arabic}
    \setcounter{page}{1}

% 正文内容
% 建议使用 \input{<文件名>} 指令引用其他文件

    \input{includes/section_01}
    \newpage
    \input{includes/section_02}
    \newpage

% 参考文献
    \addcontentsline{toc}{nonumbersection}{参考文献}
    \printbibliography[sorting=none]
    \newpage

% 致谢
    \addcontentsline{toc}{nonumbersection}{致谢}
    \begin{center}
        \zihao{3} \textbf{致谢} \\
    \end{center}

    在论文的最后我想向所有帮助支持过我的亲人、朋友、老师致以崇高的敬意和真诚的感谢,感谢你们在我三年研究生的生活中给予的生活和工作的支持。

    2017年9月,我开始了研究生生活,时间飞逝,我即将离开学校,走向社会,在此期间,我要特别感谢XX教授,是两位老师带我进入了XXXX的世界;特别感谢实验室的同学,在我碰到问题的时候伸出援手,帮助我解决问题;最后我要特别感谢我的父母,感谢你们对我学习生涯的资助,感谢你们对我未来决定的支持。

\end{document}
